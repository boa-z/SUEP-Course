% \chapter{Introduction}

% Over recent years, Neural Networks (NNs) have been successfully deployed in a wide range of applications.
% Compared to conventional Artificial Neural Networks (ANNs) which use analog values to represent activations inside the networks, 
% Spiking Neural Networks (SNNs) imitate real biological neurons and encode the activation with the timing information of neuron spikes.
% As opposed to ANNs, where the major operations are the matrix multiplication of weights and activation of network layers,
% the spiking nature of SNNs avoids complex matrix multiplications and thus they require lower computation resources and are more energy efficient.

% SNNs are built with neurons, which are connected with weighted synapses to form layers and networks.
% As the basic building block of SNNs, each spiking neuron maintains a value to represent its current state.
% Inputs to these neurons are spikes transmitted through synapses.
% The neurons collect these input spikes and integrate the corresponding weights to update their internal states.
% Whenever the state reaches a certain threshold value, the neuron is activated and outputs a spike to its successor neurons.
% After a pre-configured propagation delay, the successors receive the spike and update their states accordingly.
% Besides these common operating principles, different neuron models feature different neuron behaviors.
% One of the most commonly used neuron models is the Leaky Integrate-and-Fire (LIF) model.
% LIF neurons are inspired by the membrane voltage leaking in biological neurons, with their states decreasing over time if no input spikes present.

% However, current computer architectures are not ideally suited to executing SNNs.
% The massive parallelism inherent to an SNN, in which a large number of neurons work in a similar but simple manner, 
% requires a more parallel architecture than that provided by current Central Processing Units (CPUs).
% Although Graphics Processing Units (GPUs) can capitalize on the parallelism of SNN,
% the kernel-launch programming paradigm of current GPUs makes them ill-suited to event-driven computation,
% which is an important class of updating algorithms for SNNs (as described below in Section 4.1).
% However, Field-Programmable Gate Arrays (FPGA) can address this issue since they provide parallel processing capability and are flexibly reconfigurable to cater to different computing models.
% Furthermore, FPGA are more energy efficient than current CPUs and GPUs.

% This paper therefore proposes an FPGA-based SNN module implementation.
% By utilizing an enhanced hybrid updating algorithm, the proposed module supports up to 16 384 neurons and 16.8 million synapses with minimal on-chip resources and 0.477 W power consumption.
% We build a test platform to evaluate the proposed accelerator, on which we map an SNN model for the classification of handwritten digits in the MNIST [1] dataset.
% The evaluation results show that the classification accuracy is 97.06\% and the performance for processing neuron firing events is $6.72 \times 10^5$ events per second, resulting in 161 frames per second for the MNIST dataset.

% \rule{\linewidth}{0.5pt}

\chapter{介绍}

近年来,神经网络(NN)已成功部署在广泛的应用中。 
与使用模拟值表示网络内部激活的传统人工神经网络 (ANN) 相比,尖峰神经网络 (SNN) 模仿真实的生物神经元,
并使用神经元尖峰的时序信息对激活进行编码。 与 ANN 不同,ANN 的主要运算是权重的矩阵乘法和网络层的激活,
SNN 的尖峰特性避免了复杂的矩阵乘法,因此它们需要的计算资源更少,并且能源效率更高。

SNN 由神经元构建,神经元与加权突触连接以形成层和网络。 
作为 SNN 的基本构建块,每个尖峰神经元都维护一个代表其当前状态的值。 
这些神经元的输入是通过突触传输的尖峰。 神经元收集这些输入尖峰并整合相应的权重以更新其内部状态。 
每当状态达到某个阈值时,神经元就会被激活并向其后继神经元输出一个尖峰。 
经过预先配置的传播延迟后,后继者会收到尖峰并相应地更新其状态。 除了这些共同的操作原理之外,不同的神经元模型还具有不同的神经元行为。 最常用的神经元模型之一是泄漏积分与激发 (LIF) 模型。 LIF 神经元受到生物神经元中泄漏的膜电压的启发,如果不存在输入尖峰,它们的状态会随着时间的推移而降低。

然而,当前的计算机架构并不非常适合执行 SNN。 
SNN 固有的大规模并行性(其中大量神经元以相似但简单的方式工作)需要比当前中央处理单元 (CPU) 提供的并行架构更加并行的架构。 尽管图形处理单元 (GPU) 可以利用 SNN 的并行性,但当前 GPU 的内核启动编程范式使得它们不适合事件驱动计算,而事件驱动计算是 SNN 更新算法的一类重要内容(如下所述) 第 4.1 节)。 然而,现场可编程门阵列(FPGA)可以解决这个问题,因为它们提供并行处理能力,并且可以灵活地重新配置以满足不同的计算模型。 此外,FPGA 比当前的 CPU 和 GPU 更加节能。

因此,本文提出了一种基于FPGA的SNN模块实现。 
通过利用增强的混合更新算法,该模块以最少的片上资源和 0.477 W 功耗支持多达 16 384 个神经元和 1680 万个突触。
我们构建了一个测试平台来评估所提出的加速器,
在该平台上我们映射了一个 SNN 模型,用于对 MNIST 数据集中的手写数字进行分类。 
评估结果显示,分类准确率为 97.06\%,
处理神经元放电事件的性能为每秒 $6.72 \times 10^5$ 事件,导致 MNIST 数据集每秒处理 161 帧。

