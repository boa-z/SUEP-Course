% \chapter{Conclusion}

% In this paper, an FPGA-based SNN hardware implementation is proposed.
% The proposed module is designed based on a hybrid of the time-stepped and event-driven updating algorithms.
% An evaluation of the proposed module is carried out using the MNIST dataset with the results showing a classification accuracy of 97.06\%.
% With 0.477 W power consumption, the performance for processing neuron activation events is $6.72*10^5$ events per second,
% which results in a frame rate of 161 frames per second on MNIST dataset.
% The proposed module further identifies that memory bandwidth is the bottleneck of the system.
% To address this issue, two potential optimization techniques are discussed for FPGA-based SNN implementations.

% \rule{\linewidth}{0.5pt}

\chapter{结论}

本文提出了一种基于FPGA的SNN硬件实现方案。
所提出的模块是基于时间步进和事件驱动更新算法的混合设计的
使用 MNIST 数据集对所提出的模块进行评估,结果显示分类准确度为 97.06\%。
在 0.477 W 功耗下,处理神经元激活事件的性能为每秒 $6.72*10^5$ 个事件,这导致 MNIST 数据集上的帧率为每秒 161 帧。
所提出的模块进一步确定内存带宽是系统的瓶颈。 
为了解决这个问题,讨论了两种用于基于 FPGA 的 SNN 实现的潜在优化技术。
