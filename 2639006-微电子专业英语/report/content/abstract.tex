%%
% The SUEPThesis Template for Bachelor Graduation Thesis
%
% 上海电力大学毕业设计(论文)中英文摘要 —— 使用 XeLaTeX 编译
%
% Copyright 2020-2023 SUEPaper
%
% This work may be distributed and/or modified under the
% conditions of the LaTeX Project Public License, either version 1.3
% of this license or (at your option) any later version.
% The latest version of this license is in
%   http://www.latex-project.org/lppl.txt
% and version 1.3 or later is part of all distributions of LaTeX
% version 2005/12/01 or later.
%
% This work has the LPPL maintenance status `maintained'.
%
% The Current Maintainer of this work is Haiwen Zhang.
%%

% 中英文摘要章节
\begin{abstract}

  受真实生物神经模型的启发,尖峰神经网络 (SNN) 使用离散尖峰处理信息,
  并显示出构建低功耗神经网络系统的巨大潜力。 
  本文提出了一种基于现场可编程门阵列(FPGA)的SNN硬件实现方案。 
  它采用混合更新算法,结合了现有算法的优点,简化了硬件设计并提高了性能。 
  所提出的设计支持多达 16 384 个神经元和 1680 万个突触,但需要最少的硬件资源,
  并且实现 0.477 W 的极低功耗。
  使用 Xilinx FPGA 评估板基于所提出的设计构建了一个测试平台,我们在该平台上进行了部署 MNIST 数据集上的分类任务。 
  评估结果显示准确率为97.06\%,帧率为161帧/秒。

  
\end{abstract}
  
% 英文摘要章节
% \begin{abstractEn}

%   Inspired by real biological neural models, 
%   Spiking Neural Networks (SNNs) process information with discrete spikes and show great potential for building low-power neural network systems. 
%   This paper proposes a hardware implementation of SNN based on Field-Programmable Gate Arrays (FPGA). 
%   It features a hybrid updating algorithm, 
%   which combines the advantages of existing algorithms to simplify hardware design and improve performance. 
%   The proposed design supports up to 16 384 neurons and 16.8 million synapses but requires minimal hardware resources and archieves a very low power consumption of 0.477 W. 
%   A test platform is built based on the proposed design using a Xilinx FPGA evaluation board, 
%   upon which we deploy a classification task on the MNIST dataset. 
%   The evaluation results show an accuracy of 97.06\% and a frame rate of 161 frames per second.

% \end{abstractEn}
  