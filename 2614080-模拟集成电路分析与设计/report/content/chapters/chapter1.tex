%%
% The SUEPThesis Template for Bachelor Graduation Thesis
%
% 上海电力大学毕业设计(论文)中英文摘要 —— 使用 XeLaTeX 编译
%
% Copyright 2020-2023 SUEPaper
%
% This work may be distributed and/or modified under the
% conditions of the LaTeX Project Public License, either version 1.3
% of this license or (at your option) any later version.
% The latest version of this license is in
%   http://www.latex-project.org/lppl.txt
% and version 1.3 or later is part of all distributions of LaTeX
% version 2005/12/01 or later.
%
% This work has the LPPL maintenance status `maintained'.
%
% The Current Maintainer of this work is Haiwen Zhang.
%%

\chapter{引言选题意义}

在当今的信息技术时代,模拟CMOS电路作为集成电路设计的重要组成部分,在通信、医疗、消费电子等多个领域扮演着举足轻重的角色。随着电子设备功能的日益复杂和性能要求的不断提高,传统的模拟CMOS电路设计方法已经难以满足快速发展的市场需求。这些传统方法往往依赖于设计师的经验和试错过程,导致设计周期长、成本高,且难以保证设计的一致性和可扩展性。因此,探索新的、高效的模拟CMOS电路设计方法具有重要的现实意义。

近年来,人工智能(AI)技术,尤其是机器学习和深度学习,已经在众多领域展现出其强大的数据处理能力和模式识别能力。将这些技术应用于模拟CMOS电路设计,可以极大地提高设计效率,优化电路性能,缩短开发周期,并降低对设计师经验的依赖。AI技术在模拟电路设计中的应用,不仅能够处理复杂的设计参数和性能指标,还能通过学习历史设计数据来预测和优化新的设计,实现设计的自动化和智能化。

然而,尽管人工智能技术在模拟CMOS电路设计中展现出巨大的潜力,但其应用仍面临诸多挑战,如设计空间的巨大、设计参数与性能之间的关系复杂、以及对高精度仿真工具的依赖等。此外,如何将AI技术与现有的设计流程和工具链无缝集成,也是实现其广泛应用需要解决的问题。因此,深入研究AI在模拟CMOS电路设计中的应用,探索其优势、挑战及未来发展方向,对于推动集成电路设计领域的技术进步具有重要的科学意义和应用价值。本文旨在综述AI技术在模拟CMOS电路设计中的应用现状,分析其发展趋势,并对未来的研究方向提出展望。
