%%
% The SUEPThesis Template for Bachelor Graduation Thesis
%
% 上海电力大学毕业设计(论文)中英文摘要 —— 使用 XeLaTeX 编译
%
% Copyright 2020-2023 SUEPaper
%
% This work may be distributed and/or modified under the
% conditions of the LaTeX Project Public License, either version 1.3
% of this license or (at your option) any later version.
% The latest version of this license is in
%   http://www.latex-project.org/lppl.txt
% and version 1.3 or later is part of all distributions of LaTeX
% version 2005/12/01 or later.
%
% This work has the LPPL maintenance status `maintained'.
%
% The Current Maintainer of this work is Haiwen Zhang.
%%

\documentclass{suepthesis}

\SUEPSetup{
  cover = {
    %% 使用以下参数来自定义封面日期
    date = 2024年10月18日,
  },
  info = {
    title = 人工智能在模拟CMOS电路设计中的应用与展望,
    % 当且仅当封面题目的第一栏写不下,再使用subtitle用于扩充
    % subtitle = 问题的优化建模,
    % titleEn = {Optimal Modeling of Production and Storage Planning for WPCR},
    institution = 电子与信息工程学院,
    major = 集成电路设计与集成系统2021级,
    author = 某学生,
    studentId = 2021xxxx,
    supervisor = 某教师,
    keywords = {人工智能; 模拟CMOS电路设计; 电子设计自动化; 强化学习; 深度神经网络; 性能优化; 设计自动化流程; 设计迁移; 仿真与验证; 进化优化算法},
    % keywordsEn = {Dynamic Programming; 0-1 Programming; Analytic Hierarchy Process; Genetic Algorithm; Strategy Iteration},
  }
}

% 使用 BibLaTeX 处理参考文献
%   biblatex-gb7714-2015 常用选项
%     gbnamefmt=lowercase     姓名大小写由输入信息确定
%     gbpub=false             禁用出版信息缺失处理
\usepackage[
  backend=biber,
  style=gb7714-2015,
  gbalign=gb7714-2015,
  gbnamefmt=lowercase,
  gbpub=false
]{biblatex}     

% 参考文献引用文件位于 content/thesis.bib
\addbibresource{content/thesis.bib}

% 脚注格式
\usepackage[perpage,bottom,hang]{footmisc}

% 定义图片文件目录与扩展名
\graphicspath{{figures/}{images/}}
\DeclareGraphicsExtensions{.pdf,.eps,.png,.jpg,.jpeg}

% 使用三线表:toprule,midrule,bottomrule。
\usepackage{booktabs}

% 表格中支持跨行
\usepackage{multirow}

% 表格中数字按小数点对齐
\usepackage{dcolumn}
\newcolumntype{d}[1]{D{.}{.}{#1}}

% 使用长表格
\usepackage{longtable}

% 附带脚注的表格
\usepackage{threeparttable}

% 附带脚注的长表格
\usepackage{threeparttablex}

% 算法环境宏包
\usepackage[ruled,vlined,linesnumbered]{algorithm2e}

% 直立体数学符号
\providecommand{\dd}{\mathop{}\!\mathrm{d}}
\providecommand{\ee}{\mathrm{e}}
\providecommand{\ii}{\mathrm{i}}
\providecommand{\jj}{\mathrm{j}}

% 国际单位制宏包
\usepackage{siunitx}[=v2]

% 绘图宏包
\usepackage{tikz}
\usetikzlibrary{positioning, shapes.geometric}

% 一些文档中用到的 logo
\usepackage{hologo}
\providecommand{\XeTeX}{\hologo{XeTeX}}
\providecommand{\BibLaTeX}{\textsc{Bib}\LaTeX}

% 借用 ltxdoc 里面的几个命令方便写文档
\DeclareRobustCommand\cs[1]{\texttt{\char`\\#1}}
\providecommand\pkg[1]{{\sffamily#1}}

% 文档开始
\begin{document}

% 标题页面:如无特殊需要,本部分无需改动
\MakeCover

% % 原创性声明:如无特殊需要,本部分无需改动
% % ====== 原创性声明(LaTeX 格式)======
% \MakeOriginality
% % ====== 原创性声明(LaTeX 格式)======

% 前置页面定义
\frontmatter
% 摘要:在摘要相应的 TeX 文件处进行摘要部分的撰写
%%
% The SUEPThesis Template for Bachelor Graduation Thesis
%
% 上海电力大学毕业设计(论文)中英文摘要 —— 使用 XeLaTeX 编译
%
% Copyright 2020-2023 SUEPaper
%
% This work may be distributed and/or modified under the
% conditions of the LaTeX Project Public License, either version 1.3
% of this license or (at your option) any later version.
% The latest version of this license is in
%   http://www.latex-project.org/lppl.txt
% and version 1.3 or later is part of all distributions of LaTeX
% version 2005/12/01 or later.
%
% This work has the LPPL maintenance status `maintained'.
%
% The Current Maintainer of this work is Haiwen Zhang.
%%

% 中英文摘要章节
\begin{abstract}

  随着人工智能技术的飞速发展,其在模拟CMOS电路设计领域的应用日益广泛。模拟CMOS电路设计作为集成电路设计中的关键环节,其复杂性和对精度的高要求使得传统设计方法面临诸多挑战。本文首先介绍了将人工智能技术应用于模拟CMOS电路设计的重要性和必要性,随后详细讨论了人工智能技术在该领域的主要应用,包括自动化设计流程、性能优化、以及电路验证等。同时,本文还综述了国内外在这一领域的研究进展,并对人工智能技术在模拟CMOS电路设计中的未来发展趋势进行了展望。最后,文章提出了当前面临的挑战和潜在的研究方向,以期为相关领域的研究者和工程师提供参考。通过对现有文献的综合分析,本文旨在为模拟CMOS电路设计领域注入新的活力,并推动人工智能技术在该领域的进一步发展。
  
\end{abstract}
  

\MakeTOC

% 正文开始
\mainmatter

% ====== SUEPThesis 使用说明部分,正式写论文时请删除 ======
% \input{content/intro.tex}
% \input{content/math_and_citations.tex}
% \input{content/float.tex}
% ====== SUEPThesis 使用说明部分,正式写论文时请删除======


% ====== 论文主体 ======
%子章节为了便于查找和修改,建议通过input拆分文件
%%
% The SUEPThesis Template for Bachelor Graduation Thesis
%
% 上海电力大学毕业设计(论文)中英文摘要 —— 使用 XeLaTeX 编译
%
% Copyright 2020-2023 SUEPaper
%
% This work may be distributed and/or modified under the
% conditions of the LaTeX Project Public License, either version 1.3
% of this license or (at your option) any later version.
% The latest version of this license is in
%   http://www.latex-project.org/lppl.txt
% and version 1.3 or later is part of all distributions of LaTeX
% version 2005/12/01 or later.
%
% This work has the LPPL maintenance status `maintained'.
%
% The Current Maintainer of this work is Haiwen Zhang.
%%

\chapter{引言选题意义}

在当今的信息技术时代,模拟CMOS电路作为集成电路设计的重要组成部分,在通信、医疗、消费电子等多个领域扮演着举足轻重的角色。随着电子设备功能的日益复杂和性能要求的不断提高,传统的模拟CMOS电路设计方法已经难以满足快速发展的市场需求。这些传统方法往往依赖于设计师的经验和试错过程,导致设计周期长、成本高,且难以保证设计的一致性和可扩展性。因此,探索新的、高效的模拟CMOS电路设计方法具有重要的现实意义。

近年来,人工智能(AI)技术,尤其是机器学习和深度学习,已经在众多领域展现出其强大的数据处理能力和模式识别能力。将这些技术应用于模拟CMOS电路设计,可以极大地提高设计效率,优化电路性能,缩短开发周期,并降低对设计师经验的依赖。AI技术在模拟电路设计中的应用,不仅能够处理复杂的设计参数和性能指标,还能通过学习历史设计数据来预测和优化新的设计,实现设计的自动化和智能化。

然而,尽管人工智能技术在模拟CMOS电路设计中展现出巨大的潜力,但其应用仍面临诸多挑战,如设计空间的巨大、设计参数与性能之间的关系复杂、以及对高精度仿真工具的依赖等。此外,如何将AI技术与现有的设计流程和工具链无缝集成,也是实现其广泛应用需要解决的问题。因此,深入研究AI在模拟CMOS电路设计中的应用,探索其优势、挑战及未来发展方向,对于推动集成电路设计领域的技术进步具有重要的科学意义和应用价值。本文旨在综述AI技术在模拟CMOS电路设计中的应用现状,分析其发展趋势,并对未来的研究方向提出展望。

\chapter{三选一任务}

\section{设计目的}

任务三 频率计仿真和综合

1、按下列功能要求,使用 verilog 语言编写 RTL 代码及对应测试代码

用状态机设计一个自动转换量程的频率计的控制部分,用case语句描述。

2、仿真调试完成后,按下列要求综合,最后版图设计

按下列要求完成约束文件的编写:

\begin{itemize}
    \item 建立名为 \texttt{my\_clk} 的250Mhz的时钟;
    \item 设置时钟漂移为0.25ns;
    \item 时钟寄存器不与时钟网络连接;
    \item 设置除时钟外输入端口的表达式,类似宏定义;
    \item 设置除时钟外输入端口最大延时为1ns;
    \item 设置输出端口延时为3ns。
\end{itemize}

综合后生成timing.rpt文件,检查确保slack>0,生成.sdf文件,用于后仿真,

其中:
clk: 为同步时钟信号;
Clr: 为异步复位信号。

\section{设计思路}

目的是实现频率计的控制模块,使用状态机管理自动量程转换逻辑。通过时钟信号和异步清零信号驱动状态切换,当清零信号触发时,状态初始化为STATE\_C。状态转换根据计数溢出(Cntover)或计数偏低(Cntlow)条件进行。在不同状态下,模块设置特定的量程(range)、频率选择信号(std\_f\_sel),并根据需要控制复位信号(reset)。组合逻辑部分确保每次状态切换时输出信号与相应状态匹配,并通过case语句管理六种状态间的跳转,确保量程自动切换的逻辑满足频率计的控制需求。

% project 构造参考例程

\section{设计步骤}

\subsection{RTL 代码编写}

% show code

\begin{minted}[
    frame=lines,
    framesep=2mm,
    baselinestretch=1.2,
    bgcolor=lightgray!20,
    fontsize=\small,
    linenos
  ]{verilog}
  `timescale 1ns/1ps

  module frequency_counter_control(
      input wire Clk,
      input wire Clear,
      input wire Cntover,
      input wire Cntlow,
      output reg reset,
      output reg [1:0] std_f_sel,
      output reg [2:0] range
  );
      reg [2:0] state, next_state;
  
      // 定义状态参数
      localparam STATE_A = 3'b000, // 100K stateA
                 STATE_B = 3'b001, // 100K stateB
                 STATE_C = 3'b010, // 10K stateC
                 STATE_D = 3'b011, // 10K stateD
                 STATE_E = 3'b100, // 1K stateE
                 STATE_F = 3'b101; // 1K stateF
  
      // 时序逻辑:状态切换
      always @(posedge Clk or posedge Clear) begin
          if (Clear)
              state <= STATE_C;  // 初始化为 STATE_C
          else
              state <= next_state;
      end
  
      // 组合逻辑:状态转移和输出控制
      always @(*) begin
          // 默认值初始化,减少冗余赋值
          range = 3'b010;
          std_f_sel = 2'b01;
          reset = 1;
          next_state = state;  // 默认保持当前状态
  
          case (state)
              STATE_A: begin
                  range = 3'b000;
                  std_f_sel = 2'b00;
                  next_state = STATE_B;
              end
  
              STATE_B: begin
                  range = 3'b000;
                  std_f_sel = 2'b00;
                  reset = 0;
                  if (Cntlow)
                      next_state = STATE_C;
              end
  
              STATE_C: begin
                  range = 3'b010;
                  std_f_sel = 2'b01;
                  next_state = STATE_D;
              end
  
              STATE_D: begin
                  range = 3'b010;
                  std_f_sel = 2'b01;
                  reset = 0;
                  if (Cntover)
                      next_state = STATE_A;
                  else if (Cntlow)
                      next_state = STATE_E;
              end
  
              STATE_E: begin
                  range = 3'b100;
                  std_f_sel = 2'b11;
                  next_state = STATE_F;
              end
  
              STATE_F: begin
                  range = 3'b100;
                  std_f_sel = 2'b11;
                  reset = 0;
                  if (Cntover)
                      next_state = STATE_C;
              end
  
              default: begin
                  next_state = STATE_C;
              end
          endcase
      end
  endmodule
\end{minted}  

\subsection{测试代码编写与仿真调试}

通过搭建简单的testbench检验设计代码的正确性,tb\_frequency\_counter\_control.v文件为 verilog 测试平台代码。

% show code

\begin{figure}[H]
    \centering
    \includegraphics[width=0.8\textwidth]{images/final-task03-01.png}
    \caption{图1 仿真波形}
\end{figure}

进入仿真目录 verify/vtb/sim ,执行 make 命令,使用VCS数字逻辑仿真工具编译RTL和测试平台并进行仿真

\subsection{约束文件编写}

用 DC 工具进行编译和综合。

完成 frequency\_counter\_control.sdc 的编写后,进行综合。

% show code

\begin{minted}[
    frame=lines,
    framesep=2mm,
    baselinestretch=1.2,
    bgcolor=lightgray!20,
    fontsize=\small,
    linenos
  ]{text}
  create_clock -name my_clk -period 4.0 [get_ports Clk]

  set_clock_uncertainty 0.25 [get_clocks my_clk]
  
  set_dont_touch_network [get_clocks my_clk]
  
  set_input_delay -max 1.0 [get_ports {Clear reset Cntover Cntlow}]
  
  set_input_delay -max 1.0 [get_ports {Clear reset Cntover Cntlow}] -clock my_clk
  
  set_output_delay -max 3.0 [get_ports {range std_f_sel}] -clock my_clk
  
  set_output_delay -min 0.0 -max 3.0 [get_ports {range std_f_sel}] -clock my_clk
  
  set_dont_touch [get_ports {Clear reset Cntover Cntlow}]
\end{minted}

以上约束文件的作用分别是:

1、创建一个名为 my\_clk 的时钟信号,并指定其周期为 4.0 ns。时钟频率为 250MHz。

2、为 my\_clk 时钟设置 不确定性(uncertainty)为 0.25 ns。

3、禁止工具对 my\_clk 的时钟网络进行任何优化。

4、设置输入端口 Clear reset Cntover Cntlow 的最大输入延迟为 1.0 ns。

5、将输入延迟约束与时钟 my\_clk 关联起来

6、设置输出端口 range std\_f\_sel 的最大输出延迟为 3.0 ns,与 my\_clk 时钟同步。

7、进一步细化输出延迟,指定最小延迟为 0.0 ns,最大延迟为 3.0 ns。

8、禁止工具对这些输入端口进行优化。


1、	进入到dc路径下,启动dc,键入: \texttt{dc\_shell}

2、	在dc\_shell命令框下输入命令: \texttt{source run\_dc.tcl}
注:run\_dc.tcl 是一个tcl脚本,主要实现了Design Compiler的详细步骤。

3、	在命令栏输入 \texttt{report\_timing}

\begin{figure}[H]
    \centering
    \includegraphics[width=0.8\textwidth]{images/final-task03-02.png}
    \caption{图2 综合结果}
\end{figure}

如图所示 slack 值为正数,则表明设计满足时序要求

\subsection{Formal}

1、验证综合前后的网表逻辑等价性检查。
2、验证综合后的网表与绕线后的网表逻辑等价性检查。

分别运行 \texttt{run\_G2G.tcl} 和 \texttt{run\_R2G.tcl} 脚本,进行Formal验证。如图所示通过:

\begin{figure}[H]
    \centering
    \includegraphics[width=0.8\textwidth]{images/final-task03-11.png}
    \caption{G2G}
\end{figure}

\begin{figure}[H]
    \centering
    \includegraphics[width=0.8\textwidth]{images/final-task03-12.png}
    \caption{R2G}
\end{figure}

\subsection{版图设计}

使用TCL脚本使用innovus工具进行布局布线。

首先将 syn/dc/frequency\_counter\_control.v 到 pr/ 目录下。

编辑 run\_innovus.tcl 和 viewDefinition.tcl 文件,随后运行 \texttt{innovus -files run\_innovus.tcl}。

运行 \texttt{check\_timing -verbose > check.rpt},然后打开 check.rpt,查看时序报告。

\begin{figure}[H]
    \centering
    \includegraphics[width=0.8\textwidth]{images/final-task03-03.png}
    \caption{图4 时序报告}
\end{figure}

运行 \texttt{timeDesign -prePlace},  报出 timing 情况

\begin{figure}[H]
    \centering
    \includegraphics[width=0.8\textwidth]{images/final-task03-04.png}
    \caption{图5 时序报告}
\end{figure}

完成布局规划,通过继续运行 \texttt{run\_innovus\_step\_1\_floorplan.tcl} 进行布局布线。

1) 制定芯片面积

2) 完成 IO pin 分配

\begin{figure}[H]
    \centering
    \includegraphics[width=0.8\textwidth]{images/final-task03-05.png}
    \caption{图6 布局布线}
\end{figure}

完成电源规划,运行 \texttt{run\_innovus\_step\_2\_powerplan.tcl}。

1) 定义全局电源规划

2) 添加 Power Stripe
    
3) 添加电源 follow pin

\begin{figure}[H]
    \centering
    \includegraphics[width=0.8\textwidth]{images/final-task03-08.png}
    \caption{图7 电源规划}
\end{figure}

5. 保存数据

6. 完成plaecement

7. 检查place结果

8. CTS

9. 检查CTS结果

10. Route

\begin{figure}[H]
    \centering
    \includegraphics[width=0.8\textwidth]{images/final-task03-09.png}
    \caption{图8 布局布线}
\end{figure}

保存网表,命令:\texttt{saveNetlist route.v}
保存数据,命令:\texttt{saveDesign route.enc}
保存DEF,命令:\texttt{defOut -routing -withShield route.def}

\subsection{QRC 工具提取寄生参数}

使用 QRC 工具进行寄生参数的提取,使用 \texttt{qrc -cmd qrc.cmd},运行 qrc,如图所示:

QRC 提取运行完成:

\begin{figure}[H]
    \centering
    \includegraphics[width=0.8\textwidth]{images/final-task03-10.png}
    \caption{图9 QRC 提取结果}
\end{figure}

9. 进入到 \texttt{qrc\_test} 目录中,里面为提取出的寄生参数文件: \texttt{frequency\_counter\_control.spef}

\section{实验总结}

本任务通过Verilog编写频率计的控制部分,并使用状态机实现自动转换量程功能。完成仿真调试后,通过TCL脚本在Design Compiler中进行综合,确保时序裕度(slack)大于零,并生成.sdf文件用于后仿真。综合后的设计在Innovus工具中完成布局布线,通过时序检查和电源规划确保设计符合要求,最终生成网表和DEF文件作为输出结果。

% \chapter{Related Work}

% \section{Hardware implementation}

% Significant effort has been invested into the hardware implementation of SNNs and some designs have been presented in the form of digital [3–7] ,
% analog [8,9] , and mixed analog/digital circuits [10,11] . 
% For digital implementations, both FPGA-based systems and Application-Specific Integrated Circuit (ASIC) systems have been widely studied.

% The TrueNorth chipis one of the most well-known ASIC designs. 
% A core in the TrueNorth system contains a 256$\times$256 crossbar that implements the function of synapses and is configured to map incoming spikes to neurons. 
% By integrating 4096 such processing cores, the TrueNorth chip carries 1 million neurons and 256 million synapses. 
% The scale can be further extended by connecting multiple chips together. 
% SpiNNaker [4] is another fully custom digital system and is composed of many small ARM processors. 
% It features a custom interconnect communication scheme that is designed to be suitable for a large number of small spike-like messages and thus optimized for the communication behavior of a spike-based network architecture. 
% Like TrueNorth, SpiNNaker supports the cascading of multiple chips to form large-scale systems.

% Previous works [5–7] have also proposed several FPGAbased SNN accelerator designs. 
% BlueHive [5] supports up to 65 536 neurons and 67.1 million synapses with a multiFPGA architecture. 
% However, it implements the neurons with the complex Izhikevich modeland a low firing rate assumption, which targets biological neural network simulations and does not support real-world applications effectively. 
% Another representative design is Minituar [6] , 
% which treats the activation of neurons as events and utilizes an event-driven algorithm to update them. 
% It implements up to 65 536 LIF neurons and 16.8 million synapses. 
% Minituar faithfully models the exponential leaky process of neurons and employs on-chip Digital Signal Processors (DSPs) to carry out the fixed-point computation. 
% It also maintains a hardware event queue that requires a sorting operation for each incoming event to support spikes with delays; this increases design complexity and run-time latency.

% \rule{\linewidth}{0.5pt}

\chapter{相关工作}

\section{硬件实现}

人们在 SNN 的硬件实现上投入了大量精力,并且一些设计以数字、模拟和混合模拟/数字电路的形式呈现。 
对于数字实现,基于 FPGA 的系统和专用集成电路 (ASIC) 系统都已得到广泛研究。

TrueNorth 芯片是最著名的 ASIC 设计之一。 
TrueNorth 系统的核心包含一个 256$\times$256 的交叉开关,它实现突触的功能,并被配置为将传入的尖峰映射到神经元。 
通过集成 4096 个此类处理核心,TrueNorth 芯片可承载 100 万个神经元和 2.56 亿个突触。 
通过将多个芯片连接在一起,可以进一步扩展规模。 SpiNNaker是另一个完全定制的数字系统,由许多小型 ARM 处理器组成。 
它具有定制的互连通信方案,该方案被设计为适合大量小的尖峰状消息,从而针对基于尖峰的网络架构的通信行为进行了优化。 
与TrueNorth一样,SpiNNaker支持多个芯片级联形成大规模系统。

之前的工作也提出了几种基于 FPGA 的 SNN 加速器设计。 
BlueHive 采用 multiFPGA 架构,支持多达 65536 个神经元和 6710 万个突触。 
然而,它使用复杂的 Izhikevich 模型和低放电率假设来实现神经元,其目标是生物神经网络模拟,不能有效支持现实世界的应用。 
另一个代表性设计是Minituar ,它将神经元的激活视为事件,并利用事件驱动的算法来更新它们。 
它实现了多达 65536 个 LIF 神经元和 1680 万个突触。 
Minituar 忠实地模拟了神经元的指数泄漏过程,并采用片上数字信号处理器 (DSP) 来执行定点计算。 
它还维护一个硬件事件队列,需要对每个传入事件进行排序操作,以支持带有延迟的尖峰; 这增加了设计复杂性和运行时延迟。

% \section{Network model}

% In recent years, Deep Belief Networks (DBNs)  have been proven to be effective in a variety of domains, such as machine vision  and machine audition  . 
% DBN is a multilayered probabilistic generative model that uses a stacked structure of multiple Restricted Boltzmann Machines (RBMs). 
% Previous work  has proposed methods to convert DBNs to LIF-based spiking DBNs and explored the processing of spiking DBNs with the event-driven algorithm.

% Another studytried to solve the loss of accuracy arising in the conversion from Fully-Connected Networks (FCNs) and Convolutional Neural Networks (CNNs) to SNNs. 
% The proposed optimization techniques include using Rectified Linear Units (ReLUs) with zero bias during training to suit spiking encoding, a weight normalization method to help regulate firing rates, and a threshold balancing scheme to enable low-latency processing.

% In this paper, we use the techniques proposed in Ref. to train our SNN model and explore the efficient hardware implementation of such SNN models.

% \rule{\linewidth}{0.5pt}

\section{网络模型}

近年来,深度置信网络(DBN)已被证明在机器视觉和机器试听等多个领域中有效。 
DBN 是一种多层概率生成模型,使用多个受限玻尔兹曼机 (RBM) 的堆叠结构。 
之前的工作提出了将 DBN 转换为基于 LIF 的尖峰 DBN 的方法,并探索了事件驱动算法对尖峰 DBN 的处理。

另一项研究试图解决从全连接网络(FCN)和卷积神经网络(CNN)到 SNN 转换时出现的精度损失。 
所提出的优化技术包括在训练期间使用零偏差的整流线性单元 (ReLU) 以适应尖峰编码、帮助调节发射率的权重归一化方法以及实现低延迟处理的阈值平衡方案。

在本文中,我们使用参考文献中提出的技术训练我们的 SNN 模型并探索此类 SNN 模型的高效硬件实现。

% \chapter{System Design}

% \section{Hybrid updating algorithm}

% We use an updating algorithm that is a hybrid of the conventional time-stepped updating algorithm and the event-driven updating algorithm. 
% In this subsection, we first briefly describe these two existing algorithms and then present our hybrid.

% The time-stepped algorithm processes all of the neurons based on discrete time steps. Within each time step, the state of each neuron is updated and checked to decide whether it outputs a spike. 
% Information about these spikes is stored for use in future time steps according to their transmission delay. 
% This algorithm can waste computing resources since it schedules unnecessary operations for neurons that do not receive any input spikes. 
% The event-driven algorithm, on the other hand, processes only the activation events of neurons. 
% An event queue is used as storage for the events, and is sorted by the event timestamps. 
% After each event dequeues from the event queue, only the states of successive neurons are updated, thereby generating new events. 
% In this way, unnecessary operations are avoided. 
% Although the event-driven algorithm can be efficient, the hardware implementation of the event queue is complicated since it requires sorting the events whenever a new event enqueues.

% Therefore, we combine the time-stepped and eventdriven algorithm, as described in Algorithm 1. 
% We use multiple event queues, each of which is tagged with timestamp $Q_n$ (where n ranges from $0$ to $D-1$ where $D$ is the maximum delay allowed), to store the events to be processed after n time steps from the current time. 
% In this way, events with the same timestamp can be stored in the same queue with no sorting operation required. 
% To manage these event queues, time steps are maintained globally. 
% At each time step, the event queue with tag $Q_0$ is set to be the active queue and its events are processed. 
% Once an event queue is empty, the current time step finishes. 
% Before the next time step, the tags of all event queues decrease by one, such that $Q_1 - Q_{D-1}$ becomes $Q_0 - Q_{D-2}$ and $Q_0$ is reused in a circular manner as $Q_{D-1}$ . 
% Sorting operations are avoided in this hybrid updating algorithm, which reduces the system’s run-time latency.

% \rule{\linewidth}{0.5pt}

\chapter{系统设计}

\section{混合更新算法}

我们使用的更新算法是传统的时间步进更新算法和事件驱动更新算法的混合。
在本小节中,我们首先简要描述这两种现有算法,然后介绍我们的混合算法。

时间步长算法根据离散时间步长处理所有神经元。 
在每个时间步内,每个神经元的状态都会被更新和检查,以确定它是否输出尖峰。
有关这些尖峰的信息被存储起来,以便根据它们的传输延迟在未来的时间步中使用。
该算法可能会浪费计算资源,因为它为不接收任何输入尖峰的神经元安排不必要的操作。
另一方面,事件驱动算法仅处理神经元的激活事件。 事件队列用于存储事件,并按事件时间戳排序。
每个事件从事件队列中出队后,仅更新连续神经元的状态,从而生成新事件。 这样就避免了不必要的操作。
尽管事件驱动算法可能很高效,但事件队列的硬件实现很复杂,因为每当有新事件入队时就需要对事件进行排序。

因此,我们结合了时间步进和事件驱动算法,如算法 1 中所述。
我们使用多个事件队列,每个事件队列都标有时间戳 $Q_n$(其中 n 范围从 $0$ 到 $D-1$,其中 $D$ 是允许的最大延迟),存储从当前时间开始n个时间步后要处理的事件。 
这样,具有相同时间戳的事件就可以存储在同一个队列中,而无需进行排序操作。 为了管理这些事件队列,需要全局维护时间步长。 
在每个时间步,带有标记 $Q_0$ 的事件队列被设置为活动队列,并处理其事件。 一旦事件队列为空,当前时间步就结束。 
在下一个时间步之前,所有事件队列的标签减一,使得 $Q_1 - Q_{D-1}$ 变为 $Q_0 - Q_{D-2}$ 并且 $Q_0$ 以循环方式重用: $Q_{D-1}$ 。 
这种混合更新算法避免了排序操作,从而减少了系统的运行时延迟。

% \section{System architecture}

% The architecture of the proposed module is shown in Fig. 1.

% \begin{algorithm}
%     \caption{Hybrid updating algorithm}
%     \KwData{Event queue $Q_0, Q_1, \dots, Q_{D-1}$}
%     \For{t $\gets$ 0: $\Delta t$: T}{
%         \While{\textbf{not} $Q_0$.is\_empty()}{
%             event $\gets Q_0$.dequeue()\;
%             \For{neuron in event.successors}{
%                 neuron.update\_state\;
%                 neuron.check\_activation\;
%                 \If{neuron.is\_activated()}{
%                     new\_event $\gets$ neuron.form\_new\_event()\;
%                     delay $\gets$ new\_event.get\_delay()\;
%                     $Q_{delay}$.enqueues(new\_event)\;
%                 }
%             }
%         }
%         \For{i $\gets$ 1: D-1}{
%             $Q_{i-1} \gets Q_{i}$\;
%         }
%         $Q_{D-1} \gets Q_0$
%     }
% \end{algorithm}

% There are four main memory components, each of which has its own controller to manage its reading and writing operations. 
% The event queues submodule is the hardware implementation of the multiple event queues described above in Section 4.1. 
% The event controller submodule is in charge of managing these event queues by enqueuing generated events and dequeuing events for processing.
% The weight memory and state memory submodules are used to store weight and state data, respectively.
% The weight memory submodule is read-only, while the state controller also controls the writing back of updated states from the state updater.
% Another memory submodule is the delay memory, which is read-only and stores the delay values of different events.
% Details about the implementation of the memory components are discussed below in Section 4.3.

% The state updater carries out the main body of computation. 
% It first decays the neuron states and then sums the incoming weights to update the neuron states.
% Checks for neuron activation are then carried out to decide whether a new event is generated.
% If any neuron is activated, its neuron state is reset to a predetermined constant.
% The state updater can exploit the parallelism of SNN by updating multiple neuron states at the same time.
% The layered structure of SNN ensures that the successors of a neuron are independent of each other, which makes simultaneous updating possible.

% The execution flow is as follows. 
% The event controller receives controlling signals and values of the current time step from the system controller (which is omitted from Fig. 1 for clarity).
% It then sets the current event queue to be active and sequentially reads events from it.
% The event data is sent to the weight controller and the state controller.
% They access the weight and state memory with the event data and then send them to the state updater.

% \begin{figure}[htb]
% \begin{center}
% \includegraphics[width=0.8\textwidth]{../assets/Fig1.jpg}
% \end{center}
% \vspace{-0.1in}  
% \caption{Architecture of the proposed system.}
% \label{fig:Architecture of the proposed system.}
% \end{figure}

% If there are activation events after the state update, these events go through the delay controller to look up their delays.
% The event controller calculates the corresponding destination event queues according to the delay, and writes the events to them.

% The system works in an asynchronous manner to improve throughput.
% For example, after the event controller sends event data to the weight and state controllers, it begins to read the event queue immediately.
% When it collects the data request signal from the weight and state controllers, event data are sent again.
% Communication between other submodules is similar, through requests and responses.
% Another example is the source controller (also omitted from Fig. 1 for clarity) for the state updater.
% It contains two First-In-First-Outs (FIFOs) to hold the operands of the state updater from the weight controller and the state controller.
% In this way, although the latter two controllers have different memory access times, waiting between them is avoided, provided that the FIFOs still have space for incoming data.

% \rule{\linewidth}{0.5pt}

\section{系统架构}

所提出模块的架构如图 1 所示。(翻译版已省略)

有四个主要内存组件,每个组件都有自己的控制器来管理其读写操作。
事件队列子模块是上面 4.1 节中描述的多个事件队列的硬件实现。
事件控制器子模块负责通过将生成的事件入队和将事件出队进行处理来管理这些事件队列。
权重存储器和状态存储器子模块分别用于存储权重和状态数据。
权重存储子模块是只读的,而状态控制器还控制来自状态更新器的更新状态的写回。
另一个内存子模块是延迟内存,它是只读的,存储不同事件的延迟值。
有关存储器组件实现的详细信息将在下面第 4.3 节中讨论。

状态更新器执行计算的主体。
它首先衰减神经元状态,然后对输入权重求和以更新神经元状态。
然后执行神经元激活检查以确定是否生成新事件。
如果任何神经元被激活,其神经元状态将重置为预定常数。
状态更新器可以通过同时更新多个神经元状态来利用 SNN 的并行性。
SNN的分层结构保证了神经元的后继者彼此独立,这使得同步更新成为可能。

执行流程如下。
事件控制器从系统控制器(为了清楚起见,在图1中省略了)接收控制信号和当前时间步长的值。
然后,它将当前事件队列设置为活动状态,并依次从中读取事件。
事件数据被发送到权重控制器和状态控制器。
他们使用事件数据访问权重和状态存储器,然后将它们发送到状态更新器。

如果状态更新后有激活事件,这些事件将通过延迟控制器查找它们的延迟。
事件控制器根据延迟计算出对应的目标事件队列,并将事件写入其中。

系统以异步方式工作以提高吞吐量。
例如,事件控制器向权重控制器和状态控制器发送事件数据后,立即开始读取事件队列。
当它从重量和状态控制器收集数据请求信号时,事件数据被再次发送。
其他子模块之间的通信类似,通过请求和响应。
另一个例子是状态更新器的源控制器(为了清楚起见,图 1 中也省略了)。
它包含两个先进先出(FIFO),用于保存来自权重控制器和状态控制器的状态更新器的操作数。
这样,虽然后两个控制器具有不同的存储器访问时间,但只要 FIFO 仍有空间容纳传入数据,就可以避免它们之间的等待。

% \section{Implementation}

% We use signed 16-bit fixed-point numbers to represent the weights and neuron states. 
% The maximum number of neurons is set to 16 384, which results in a 14-bit index for each neuron. 
% The maximum number of neurons in one layer is 1024, with fully-connected synapses supported. 
% This means that the maximum number of synapses is 16.8 million. 
% The maximum delay is set to 16, which is adequate according to previous research.

% The proposed module needs to store up to 32 MB of weight data. 
% Since the amount of on-chip Block RAM (BRAM) of FPGA is often limited, it is impractical to store all of the weights on-chip. 
% Therefore, we use external Double Data Rate (DDR) memory to store all of the weights, while implementing all of the other memory modules with BRAM. 
% Considering that for each event the corresponding weights are always of the same group, we store these weights in consecutive spaces in the external memory. 
% With this mapping, the burst read feature of the DDR memory can be exploited to optimize the latency of memory access.

% For the event queues, we implement 16 FIFOs with BRAM that can be accessed separately with a 4-bit address. 
% To implement the hybrid updating algorithm, the event controller always reads the FIFO with the lower four bits of the current time register.
% After a new event is generated by the state updater and the delay is obtained from the delay controller, the delay value is added to the current time and the lower four bits of the result are used to select the correct FIFO to which to write the data.

% The operations of the weight updater are mainly the addition of weights and neuron states and the comparison of updated states with the threshold parameter. The decay of neurons is also carried out by the weight updater. For simplicity, the exponential computation is implemented with the subtraction of a constant. To fully utilize the memory access bandwidth, 32 adders and 32 comparators are instantiated in the weight updater. Further analysis in Section 5.4 below shows that neither the throughput of the weight updater nor its consumption of hardware resources is limiting factors of the proposed module.

% \rule{\linewidth}{0.5pt}

\section{执行}

我们使用带符号的 16 位定点数来表示权重和神经元状态。 
神经元的最大数量设置为 16 384,这导致每个神经元的索引为 14 位。 一层最大神经元数量为1024个,支持全连接突触。 
这意味着突触的最大数量为 1680 万个。 最大延迟设置为 16,根据之前的研究这已经足够了。

建议的模块需要存储最多 32 MB 的重量数据。 由于 FPGA 的片上 Block RAM (BRAM) 的数量通常是有限的,因此将所有权重存储在片上是不切实际的。 
因此,我们使用外部双倍数据速率 (DDR) 存储器来存储所有权重,同时使用 BRAM 实现所有其他存储器模块。 
考虑到每个事件对应的权重始终属于同一组,我们将这些权重存储在外部存储器的连续空间中。 
通过这种映射,可以利用 DDR 存储器的突发读取功能来优化存储器访问的延迟。

对于事件队列,我们使用 BRAM 实现 16 个 FIFO,可以使用 4 位地址单独访问。 
为了实现混合更新算法,事件控制器总是读取 FIFO 中当前时间寄存器的低四位。 
状态更新器生成新事件并从延迟控制器获取延迟后,将延迟值与当前时间相加,结果的低四位用于选择要写入数据的正确 FIFO 。

权重更新器的操作主要是权重和神经元状态的相加以及更新后的状态与阈值参数的比较。 
神经元的衰减也是由权重更新器执行的。 为了简单起见,指数计算是通过减去一个常数来实现的。 
为了充分利用存储器访问带宽,权重更新器中实例化了 32 个加法器和 32 个比较器。 
下面第 5.4 节中的进一步分析表明,权重更新器的吞吐量及其对硬件资源的消耗都不是所提出模块的限制因素。


% ====== 论文主体 ======

% 后置部分
\backmatter

% 参考文献:如无特殊需要,参考文献相应的 TeX 文件无需改动,添加参考文献请使用 BibTeX 的格式
%   添加至 misc/ref.bib 中,并在正文的相应位置使用 \cite{xxx} 的格式引用参考文献
%%
% The SUEPThesis Template for Bachelor Graduation Thesis
%
% 上海电力大学毕业设计(论文)中英文摘要 —— 使用 XeLaTeX 编译
%
% Copyright 2020-2023 SUEPaper
%
% This work may be distributed and/or modified under the
% conditions of the LaTeX Project Public License, either version 1.3
% of this license or (at your option) any later version.
% The latest version of this license is in
%   http://www.latex-project.org/lppl.txt
% and version 1.3 or later is part of all distributions of LaTeX
% version 2005/12/01 or later.
%
% This work has the LPPL maintenance status `maintained'.
%
% The Current Maintainer of this work is Haiwen Zhang.
%%

\begin{bibprint}

% 在使用时,请删除/注释上方示例内容,并启用下方语句以输出所有的参考文献
\printbibliography[heading=none]
\end{bibprint}

% 附录:在附录相应的 TeX 文件处进行附录部分的撰写
%%
% The SUEPThesis Template for Bachelor Graduation Thesis
%
% 上海电力大学毕业设计(论文)中英文摘要 —— 使用 XeLaTeX 编译
%
% Copyright 2020-2023 SUEPaper
%
% This work may be distributed and/or modified under the
% conditions of the LaTeX Project Public License, either version 1.3
% of this license or (at your option) any later version.
% The latest version of this license is in
%   http://www.latex-project.org/lppl.txt
% and version 1.3 or later is part of all distributions of LaTeX
% version 2005/12/01 or later.
%
% This work has the LPPL maintenance status `maintained'.
%
% The Current Maintainer of this work is Haiwen Zhang.
%%

\begin{appendices}

% 根据本课程所讲内容,谈谈学习心得与体会,(200-300字)。
% 根据上传的PPT所讲内容,编写一份模拟集成电路分析与设计课程学习心得与体会,字数要求200-300字。
    
\section{学习心得与体会}

通过学习模拟集成电路分析与设计课程,我对模拟电路的基础知识和设计技巧有了更深入的理解。主讲人乐应波老师深入浅出地讲解了MOSFET的物理基础、小信号模型、电流镜与偏置技术,以及差动放大器等关键概念。特别是在学习共源级、源跟随器和共栅级等基本放大器结构时,我深刻体会到了设计中的权衡和优化的重要性。课程中的实例分析和设计流程讲解,让我对模拟集成电路的实际设计有了更加清晰的认识。此外,课程还涉及了模拟集成电路设计的挑战,如功耗、速度、噪声容限等,这些内容让我意识到了在设计过程中需要考虑的多方面因素。总的来说,这门课程极大地提升了我的专业素养,为我未来的学习和研究打下了坚实的基础。

\end{appendices}

\end{document}
