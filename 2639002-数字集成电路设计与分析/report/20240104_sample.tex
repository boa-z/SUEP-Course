\documentclass[lang=cn,10pt]{elegantbook}

\title{\LaTeX{} 2639002.01实验报告课程论文模板}
% \subtitle{Elegant\LaTeX{} 经典之作}

\author{\color{red}姓名+学号}
\institute{Elegant\LaTeX{} Program}
\date{2023年11月13日}
\version{4.3}
% \bioinfo{摘要}{\color{red}针对XX问题,本论文利用支持向量机/图分类器(请同学们二选一)对该问题涉及的XX数据进行二分类,实验结果为XX。该实验结果表明XX。}

\extrainfo{不要以为抹消过去,重新来过,即可发生什么改变。—— 比企谷八幡}

\setcounter{tocdepth}{3}

\logo{logo-blue.png}
\cover{cover.jpg}

% 本文档命令
\usepackage{array}
\newcommand{\ccr}[1]{\makecell{{\color{#1}\rule{1cm}{1cm}}}}

% 额外package
\usepackage{xeCJK} % 在公式里输入中文
\xeCJKsetup{CJKmath=true} % 在公式里输入中文
\usepackage[makeroom]{cancel} % 删去符号,X

% 修改标题页的橙色带
\definecolor{customcolor}{RGB}{32,178,170}
\colorlet{coverlinecolor}{customcolor}

\def\0{{\mathbf 0}}
\def\1{{\mathbf 1}}

\def\a{{\mathbf a}}
\def\b{{\mathbf b}}
\def\c{{\mathbf c}}
\def\d{{\mathbf d}}
\def\e{{\mathbf e}}
\def\f{{\mathbf f}}
\def\g{{\mathbf g}}
\def\h{{\mathbf h}}
\def\i{{\mathbf i}}
\def\j{{\mathbf j}}
\def\k{{\mathbf k}}
\def\l{{\mathbf l}}
\def\m{{\mathbf m}}
\def\n{{\mathbf n}}
\def\o{{\mathbf o}}
\def\p{{\mathbf p}}
\def\q{{\mathbf q}}
\def\r{{\mathbf r}}
\def\s{{\mathbf s}}
\def\t{{\mathbf t}}
\def\u{{\mathbf u}}
\def\v{{\mathbf v}}
\def\w{{\mathbf w}}
\def\x{{\mathbf x}}
\def\y{{\mathbf y}}
\def\z{{\mathbf z}}

\def\A{{\mathbf A}}
\def\B{{\mathbf B}}
\def\C{{\mathbf C}}
\def\D{{\mathbf D}}
\def\E{{\mathbf E}}
\def\F{{\mathbf F}}
\def\G{{\mathbf G}}
\def\F{{\mathbf F}}
\def\H{{\mathbf H}}
\def\I{{\mathbf I}}
\def\J{{\mathbf J}}
\def\K{{\mathbf K}}
\def\L{{\mathbf L}}
\def\M{{\mathbf M}}
\def\N{{\mathbf N}}
\def\O{{\mathbf O}}
\def\P{{\mathbf P}}
\def\Q{{\mathbf Q}}
\def\R{{\mathbf R}}
\def\S{{\mathbf S}}
\def\T{{\mathbf T}}
\def\U{{\mathbf U}}
\def\V{{\mathbf V}}
\def\W{{\mathbf W}}
\def\X{{\mathbf X}}
\def\Y{{\mathbf Y}}
\def\Z{{\mathbf Z}}


\def\rE{{\text{E}}}
\def\rPr{{\text{Pr}}}
\def\Tr{{\text{Tr}}}
% \def\vec{{\textbf{}}}
\def\ie{{\textit{i.e.}}}
\def\eg{{\textit{e.g.}}}


\def\cA{{\mathcal A}}
\def\cB{{\mathcal B}}
\def\cC{{\mathcal C}}
\def\cD{{\mathcal D}}
\def\cE{{\mathcal E}}
\def\cF{{\mathcal F}}
\def\cG{{\mathcal G}}
\def\cH{{\mathcal H}}
\def\cI{{\mathcal I}}
\def\cK{{\mathcal K}}
\def\cL{{\mathcal L}}
\def\cM{{\mathcal M}}
\def\cN{{\mathcal N}}
\def\cO{{\mathcal O}}
\def\cP{{\mathcal P}}
\def\cQ{{\mathcal Q}}
\def\cR{{\mathcal R}}
\def\cS{{\mathcal S}}
\def\cT{{\mathcal T}}
\def\cU{{\mathcal U}}
\def\cV{{\mathcal V}}
\def\cW{{\mathcal W}}
\def\cX{{\mathcal X}}
\def\cY{{\mathcal Y}}
\def\cZ{{\mathcal Z}}


\def\bphi{{\pmb{\phi}}}
\def\bpsi{{\pmb{\psi}}}
\def\bphi{{\pmb{\phi}}}
\def\bpsi{{\pmb{\psi}}}
\def\balpha{{\boldsymbol \alpha}}
\def\bbeta{{\boldsymbol \beta}}
\def\bvarphi{{\boldsymbol \varphi}}
\def\bPhi{{\boldsymbol \Phi}}


\def\bcdot{{\;\boldsymbol{\cdot}\;}}

\def\ie{{\textit{i.e.}}}
\def\eg{{\textit{e.g.}}}

\begin{document}

\maketitle
\frontmatter

\tableofcontents

\mainmatter

\chapter{摘要(动机、目前相似的产品、自己的和相似的优缺点、本次报告所做的核心内容、通过本次报告发现了什么、遇到了什么问题、怎么解决的、未来可以做哪些工作,150-300字)}

\chapter{引言(动机、目前相似的产品、自己的和相似的优缺点、本次报告所做的核心内容、通过本次报告发现了什么、遇到了什么问题、怎么解决的、本次报告的基本结构,500-1000字)}

\chapter{任务模块总览(说明整体任务的目标,以及简要说明每一个任务模块的功能。)}

说明整体任务的目标,以及简要说明每一个任务模块的功能。

\chapter{任务模块详细内容、任务模块功能描述、任务模块运行结果。}

详细展示任务模块详细内容、任务模块功能描述、任务模块运行结果。例如:

\section{任务模块1:721}

\subsection{任务模块1的任务详细内容}

(a) Plot a graph of the noise margins of the CMOS
inverter (similar to Fig. 7.9) for VDD = 3.3 V,
VT N = 0.75 V, and VT P = −0.75 V. (b) Repeat
for VDD = 2.0 V, VT N = 0.50 V, and VT P =
−0.50 V.

\cite{ltc7890module}

\subsection{任务模块1的功能描述}

\begin{lstlisting}[language=Python, caption=721 CMOS noise margin]
clear;clc;close all;
KR=0:0.001:12;
n_sample=length(KR);
% VDD=3.3; %
% VTN=0.75; %
% VTP=-0.75; 
VDD=2.5;
VTN=0.5;
VTP=-0.5;

VIL=2*sqrt(KR)*(VDD-VTN+VTP)./((KR-1).*sqrt(KR+3))-(VDD-KR*VTN+VTP)./(KR-1);

VOH=((KR+1).*VIL+VDD-KR*VTN-VTP)/2;

VIH=2*KR*(VDD-VTN+VTP)./((KR-1).*sqrt(1+3*KR))-(VDD-KR*VTN+VTP)./(KR-1);

VOL=((KR+1).*VIH-VDD-KR*VTN-VTP)./(2*KR);

NML=VIL-VOL;

NMH=VOH-VIH;

figure(1);
plot(KR,NML,'r','DisplayName','NM_L');hold on;
plot(KR,NMH,'b','DisplayName','NM_H');
grid on;
xlabel('K_R=0:0.001:12');
ylabel('Noise Margin');
title('CMOS 噪声容限');
xlim([min(KR) max(KR)]);
ylim([min([min(NML) min(NMH)]) max([max(NML) max(NMH)])]);
legend;

\end{lstlisting}

\subsection{任务模块1的运行结果}

\cite{ltc7890module}

\begin{figure}[htb]
\begin{center}
\includegraphics[width=6cm,trim=0in 0in 0in 0in, clip]
{image/CMOS_noise_margin.pdf}
\end{center}
\vspace{-0.1in}  
\caption{CMOS noise margin.}
\label{fig:CMOSnoisemargin}
\end{figure}

\section{任务模块2:720}

\subsection{任务模块2的任务详细内容}

\cite{st1229}.

\cite{9929278}.

Use SPICE to plot the VTC for a CMOS inverter with (W/L)N = 2/1, (W/L)P = 5/1,
VDD = 3.3 V, VSS = 0 V, VT N = 0.75 V, and
VT P = −0.75 V. Repeat if the threshold voltages
are mismatched with values VT N = 0.85 V and
VT P = 0.65 V. Repeat for (W/L)N = 2/1 and
(W/L)P = 4/1 with the original threshold voltages. 
Plot the three curves on one graph.

\begin{lstlisting}[language=Python, caption=]
\end{lstlisting}

\subsection{任务模块2的功能描述}

\begin{equation*}
\mathrm{.tran}\;\mathrm{1m}\;\mathrm{steady}\;\mathrm{startup}
\end{equation*}

\begin{lstlisting}[language=Python, caption=transient response]
.tran 1m steady startup
\end{lstlisting}

\begin{figure}[htb]
\begin{center}
\includegraphics[width=6cm]{image/2023-10-31-720.png}
\end{center}
\vspace{-0.1in}  
\caption{VTC3.}
\label{fig:VTC3}
\end{figure}

\subsection{任务模块2的运行结果}

\begin{figure}[htb]
\begin{center}
\includegraphics[width=6cm]{image/2023-10-30-720.pdf}
\end{center}
\vspace{-0.1in}  
\caption{VTC3.}
\label{fig:VTC3_results}
\end{figure}

\section{任务模块3:}

\subsection{任务模块3的任务详细内容}

Simulate the VTC for a CMOS inverter with Kn =
2.5K p. Find the input voltage for which vO = vI
and compare to the value calculated by hand. Use
VDD = 2.5 V.

\subsection{任务模块3的功能描述}

\begin{lstlisting}[language=Python, caption=输出瞬态响应]
.tran 2.5m steady startup
\end{lstlisting}

\begin{equation*}
\mathrm{.tran\;2.5m\;steady\;startup}
\end{equation*}

\subsection{任务模块3的运行结果}

\chapter{总结(在本次任务仿真过程中,发现了哪些问题,遇到了哪些困难,如何解决的,等等。)}

在本次任务仿真过程中,发现了哪些问题,遇到了哪些困难,如何解决的,等等。

\printbibliography[heading=bibintoc, title=\ebibname]


% [1] 
% \href{https://blog.csdn.net/qq_41311604/article/details/103428503}{二元分类(Binary Classfication)与logistic回归-CSDN博客}.

% [2] \href{https://blog.csdn.net/the_ZED/article/details/128515334?ops_request_misc=&request_id=&biz_id=102&utm_term=\%E6\%9C\%BA\%E5\%99\%A8\%E5\%AD\%A6\%E4\%B9\%A0\%E7\%BA\%BF\%E6\%80\%A7\%E5\%9B\%9E\%E5\%BD\%92&utm_medium=distribute.pc_search_result.none-task-blog-2~all~sobaiduweb~default-5-128515334.142^v96^pc_search_result_base9&spm=1018.2226.3001.4187}{【机器学习】线性回归(理论)\_机器学习线性回归-CSDN博客} .

% [3] \href{https://github.com/tonio73/data-science/tree/master}{github.com} .

% [4]\href{https://blog.csdn.net/qq_42192693/article/details/121164645?ops_request_misc=\%257B\%2522request\%255Fid\%2522\%253A\%2522170365538016800225578043\%2522\%252C\%2522scm\%2522\%253A\%252220140713.130102334..\%2522\%257D&request_id=170365538016800225578043&biz_id=0&utm_medium=distribute.pc_search_result.none-task-blog-2~all~top_positive~default-1-121164645-null-null.142^v98^pc_search_result_base9&utm_term=\%E6\%94\%AF\%E6\%8C\%81\%E5\%90\%91\%E9\%87\%8F\%E6\%9C\%BA&spm=1018.2226.3001.4449}{机器学习:支持向量机(SVM)-CSDN博客} 
\end{document}
